%!TEX root = main.tex

\section*{Aufgabe 2 Euler-Formeln}

\subsection*{a)}

Ein einzelner Flap verändert $\Delta$ wie folgt:
\begin{eqnarray}
E_{I}^+ &=& E_{I} +1 \\
    E_{B}^+ &=& E_{B} +1 \\
    V_{I}^+ &=& V_{I} \\
    V_{B}^+ &=& V_{B} +1\\
    F^+ &=& F +1
\end{eqnarray}
Setzen wir das nun in die Euler Formel ein erhalten wir ausgehend von einem korrekten Netz:
\begin{eqnarray}
    &&(V_{I}^+ + V_B^+) - (E_I^+ + E_B^+) + F^+\\
    &=& (V_{I} + V_B + 1) - (E_I + 1 + E_B + 1) + F +1 \\
    &=&(V_{I} + V_B ) - (E_I +E_B ) + F 
\end{eqnarray}

Womit wir gezeigt haben, dass ein einzelner Flap das Netz korrekt lässt.

Ein einzelner Fill verändert $\Delta$ wie folgt:
\begin{eqnarray}
E_{I}^+ &=& E_{I} +2 \\
    E_{B}^+ &=& E_{B} -1 \\
    V_{I}^+ &=& V_{I} +1\\
    V_{B}^+ &=& V_{B} -1\\
    F^+ &=& F +1
\end{eqnarray}


Setzen wir das nun in die Euler Formel ein erhalten wir ausgehend von einem korrekten Netz:
\begin{eqnarray}
    &&(V_{I}^+ + V_B^+) - (E_I^+ + E_B^+) + F^+  \\
    &=&(V_{I} +1+ V_B - 1) - (E_I + 2 + E_B - 1) + F +1 \\
    &=& (V_{I} + V_B ) - (E_I +E_B ) + F = 1
\end{eqnarray}


Womit wir gezeigt haben, dass ein einzelner Fill das Netz korrekt lässt.

Die Formeln für $E_I, E_B, V_I$ und $V_B$ lauten nun:
\begin{eqnarray}
E_{I}^+ &=& F_l + 2F_i \\
    E_{B}^+ &=& 3 + F_l - F_i  \\
    V_{I}^+ &=& F_i\\
    V_{B}^+ &=& 3 + F_l - F_i\\
    F^+ &=& 1 + F_i + F_l
\end{eqnarray}

Davon ausgehend können wir schreiben:

\begin{eqnarray}
    E_I&=& 2 f_i + F_l \\
    &=& E_B-3+3F_I \\
    &=& 3V_I + E_B-3
\end{eqnarray}

und mit $F_l = E_B - 3+F_i$ dann:
\begin{eqnarray}
    F &=& 1 + V_I + E_B-3 + F_i \\
    &=& 2 V_I + E_B-2 = N
\end{eqnarray}

\subsection*{b)}
$V_B = HE_B$: Alle äußeren Knoten haben eine Halbkante.

$E_B = HE_B$: Alle äußeren Kanten haben eine Halbkante.

$HE_I = 2E_I$: Alle inneren Kanten haben 2 Halbkanten.

\begin{eqnarray}
    N = \frac{1}{3} HE &=& 1 -V +E \\
    &=& 1 - (V_I+V_B) + (E_I + E_B) \\
    &=& 1 - (V_I+HE_B) + (\frac{1}{2}HE_I + E_B) \\
    &=& 1- V_I+E_B - HE_B + \frac{1}{2} HE_I\\
    \frac{1}{3} HE_I + \frac{1}{3} HE_B &=& 1- V_I+E_B - HE_B + \frac{1}{2} HE_I\\
    -\frac{1}{6} HE_I - \frac{2}{3} HE_B &=& 1- V_I- HE_B\\
    -\frac{1}{6} HE_I + \frac{1}{3} HE_B &=& 1- V_I \\
    \frac{1}{3} HE_I - \frac{2}{3} HE_B &=& 2 V_I - 2\\
    \frac{1}{3} HE_I + \frac{1}{3} HE_B &=& 2 V_I + E_B - 2\\
    N&=& 2 V_I + E_B - 2
\end{eqnarray}

\subsection*{c)}
\subsubsection*{i.}
$V-E+F = 2 $
\subsubsection*{ii.}
$\frac{2}{n}E=F$
\subsubsection*{iii.}
$\frac{2}{m}E=V$
\subsubsection*{iv.}
$\frac{2}{m}E - E +\frac{2}{n}E =2$
\subsubsection*{iv.}
Wir können iv. umschreiben zu:
\begin{eqnarray}
    \frac{2}{E} &=& \frac{2}{m} - 1 + \frac{2}{n} \\
    \frac{1}{E} + \frac{1}{2} &=& \frac{1}{m}+ \frac{1}{n} \\
\end{eqnarray}

da $E \geq 3$ wissen wir $\frac{1}{m}+ \frac{1}{n} > \frac{1}{2}$.

Dies ist erfüllt für $(n,m) = {(3,3), (3,4), (4,3), (5,3), (3,5)}$. Es kann keine weiteren geben, da $n \geq 3$ und$m \geq 3$ und für größere Werte als 5 für $n$ oder $m$ gilt: $\frac{1}{m}+ \frac{1}{n} \leq \frac{1}{2}$.






















%!TEX root = main.tex

\section*{Aufgabe 3 Interpolation mit Bezier-Dreiecken}
Die Baryzentrischen Koordinaten von $v_i$ bezueglich $T_i$ sind auf Grund der Flaechenverhaeltnisse jeweils $3, -1, -1$.\\
Fuer die folgene Aufteilung ergeben sich die gesuchten Punkte aus
\begin{align}
 	x_0 = -x_2 + 3x_3 - x_1 \\
 	d_1 = -d_2 + 3x_2 - d_3 \\
 	d_3 = -d_4 + 3x_1 - d_5 \\
 	d_5 = -d_6 + 3x_0 - d_1 
\end{align}

\begin{center}
\begin{pspicture}(-1,-1)(6,6)
	 
	\psline[linewidth=1 pt]{-}(0,0)(6,0) 
    \psline[linewidth=1 pt]{-}(0,0)(3,5) 
    \psline[linewidth=1 pt]{-}(3,5)(6,0) 
	\psline[linewidth=0.5 pt]{-}(0,0)(3,2) 
	\psline[linewidth=0.5 pt]{-}(3,5)(3,2) 
	\psline[linewidth=0.5 pt]{-}(6,0)(3,2) 
	\psdots[linewidth= 2pt](3,2)(2,1.333)(4,1.333)(3,3)
	\psdots[linewidth= 2pt, dotstyle=square](3,1)(2,2)(4,2)(1,0.666)(5,0.6666)(3,4)
    \uput[225](0,0){$v_0$}
    \uput[225](3,5.5){$v_2$}
    \uput[225](6,0){$v_1$}
    \uput[225](3,1.8){$\tilde{v}$}

	\uput[225](3,2){$x_3$}
	\uput[225](2,1.333){$x_0$}
	\uput[225](4,1.333){$x_1$}
	\uput[225](3,3){$x_2$}
	
	\uput[225](3,1){$d_5$}
	\uput[225](2,2){$d_1$}
	\uput[225](4,2){$d_3$}
	\uput[225](1,0.666){$d_6$}
	\uput[225](5,0.666){$d_4$}
	\uput[225](3,4){$d_2$}
\end{pspicture}
\end{center}
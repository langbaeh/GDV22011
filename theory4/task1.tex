%!TEX root = main.tex

\section*{Aufgabe 1 Implizite Oberflächen}

Einsetzen des Sichtstrahls in die Funktion ergibt $ft(t) = t^3+8t-4t^2-2$




Die zu bearbeitende Intervalle nennen wir $I = \{[t_0, t_1],\dots, [t_{n-1}, t_{n}] \}$ und die Intervalle die refined werden $R = \{[r_0, r_1],\dots, [r_{m-1}, r_{m}] \}$

$I_0 = \{[ 0 ,4 ]\}$

$I_1 = \{[0,2],
     [2    ,4 ]\}$

$I_2 = \{[ 0, 1],
     [1   , 2],
     [2   ,  4 ]\}$


$I_3 = \{[ 0  ,  0.5000],
    [0.5000  ,  1.0000],
    [1.0000  ,  2.0000],
    [2.0000  ,  4.0000 ]\}$

\hspace{1.0cm}

Nun wird das Intervall $[0, 0.5]$ refined:

$R_0 = \{[ 0 , 0.2500 ],
    [0.2500  ,0.5000 ] \}$
    

$[ 0 , 0.2500 ]$ wird überprüft: es kann keine Nullstelle enthalten sein.

$R_1 = \{[0.2500, 0.5000] \}$

$R_2 = \{[0.2500, 0.3750] \}$

$R_3 = \{[0.2500,0.3125] \}$

$R_4 = \{[0.2500, 0.2812] \}$

Damit ist ein Intervall gefunden, dessen Grenzen die gleiche erste Nachkommastelle haben. Einsetzen ergibt: $ft(0.25) = -0.2344$.



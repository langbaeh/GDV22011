%!TEX root = main.tex


\section*{Aufgabe 3 B-Splines}

\subsection*{a)} % (fold)
\label{sub:a_}

Anhand des Dreiecksschemas sieht man schnell, dass nur $ B^{3}_{-2}(t), B^{2}_{-2}(t), B^{2}_{-1}(t), B^{1}_{-1}(t),$ $B^{1}_{0}(t) , B^{0}_{0}(t), B^{0}_{1}(t)$  benoetig werden.

\begin{align}
	B^{0}_{0}(t) &= 1 ~;~ t \in [0,1] \\
	B^{0}_{1}(t) &= 1 ~;~ t \in [1,2] \\
	B^{1}_{-1}(t) &= 1-t ~;~ t \in [0,1] \\
	B^{1}_{0}(t) &= \bigg \{ \begin{matrix}
							t ~;~ t\in [0,1]\\
							2-t ~;~ t\in [1,2]\\
	\end{matrix}\\ 
	B^{2}_{-2}(t) &= (1-t)^2 ~;~ t\in [1,0]\\
	B^{2}_{-1}(t) &= \bigg \{ \begin{matrix}
							2t-1.5t^2 ~;~ t\in [0,1]\\
							2-2t + 0.5t^2 ~;~ t\in [1,2]\\
	\end{matrix}\\
	B^{3}_{-2}(t) &= \bigg \{ \begin{matrix}
							3t-3.5t^2-0.25t^3 ~;~ t\in [0,1]\\
							2-3t+1.5t^2-0.25t^3 ~;~ t\in [1,2]\\
	\end{matrix}
\end{align}

%!TEX root = main.tex

\begin{centering}

\psset{linewidth=1.2\pslinewidth}
\psset{xAxisLabel={x},
       yAxisLabel={y},
       labelFontSize=\displaystyle,
       xticksize=0 4,
       yticksize=0 6,
       ticklinestyle=dashed
	}
\begin{psgraph}[Ox=0, Oy=0, Dy=1,dy=1,dx=1,Dx=1]{->}(0,0)(0,0)(3,2){6cm}{4cm}
\end{psgraph}

\end{centering}

%      \psaxes[labels=none,Ox=1,subticks=5,ticksize=0pt -4pt](3,5)(1,0)(3,5)


~\vspace{25pt}
%!TEX root = main.tex

\begin{centering}

\psset{linewidth=1.2\pslinewidth}
\psset{xAxisLabel={x},
       yAxisLabel={y},
       labelFontSize=\displaystyle,
       xticksize=0 4,
       yticksize=0 6,
       ticklinestyle=dashed
	}
\begin{psgraph}[Ox=0, Oy=0, Dy=1,dy=1,dx=1,Dx=1]{->}(0,0)(0,0)(3,2){6cm}{4cm}
\end{psgraph}

\end{centering}

%      \psaxes[labels=none,Ox=1,subticks=5,ticksize=0pt -4pt](3,5)(1,0)(3,5)


~\vspace{25pt}
%!TEX root = main.tex

\begin{centering}

\psset{linewidth=1.2\pslinewidth}
\psset{xAxisLabel={x},
       yAxisLabel={y},
       labelFontSize=\displaystyle,
       xticksize=0 4,
       yticksize=0 6,
       ticklinestyle=dashed
	}
\begin{psgraph}[Ox=0, Oy=0, Dy=1,dy=1,dx=1,Dx=1]{->}(0,0)(0,0)(3,2){6cm}{4cm}
\end{psgraph}

\end{centering}

%      \psaxes[labels=none,Ox=1,subticks=5,ticksize=0pt -4pt](3,5)(1,0)(3,5)



% subsection a_ (end)

\subsection*{b)} % (fold)
\label{sub:b_}

\begin{align}
	\lambda^{[1]}_{0,1}(t) &=  \frac{3-t}{2} ~~ &\lambda^{[1]}_{0,0}(t) &=  \frac{3-t}{3}\\
	\lambda^{[1]}_{0,-1}(t) &= \frac{2-t}{2} ~~ &\lambda^{[1]}_{1,1}(t) &=  \frac{t-1}{2}\\
	\lambda^{[1]}_{1,0}(t) &=  \frac{t}{3}   ~~ &\lambda^{[1]}_{1,-1}(t) &= \frac{t}{2}\\
	\lambda^{[2]}_{0,1}(t) &=  \frac{3-t}{2} ~~ &\lambda^{[2]}_{0,0}(t) &=  \frac{2-t}{2}\\
	\lambda^{[2]}_{1,1}(t) &=  \frac{t-1}{2} ~~ &\lambda^{[2]}_{1,0}(t) &=  \frac{t}{2}\\
	\lambda^{[3]}_{0,1}(t) &=  2-t           ~~ &\lambda^{[3]}_{1,1}(t) &=  t-1
\end{align}
\begin{align}
	b^{[0]}_{-2} &= \vect{4 \\ 4} ~~ &b^{[0]}_{-1} &= \vect{0 \\ 4}\\
	b^{[0]}_{0} &=  \vect{0 \\ 0} ~~ &b^{[0]}_{1} &=  \vect{4 \\ 0}\\
	b^{[1]}_{1} &=  \vect{1 \\ 4} ~~ &b^{[1]}_{0} &=  \vect{0 \\ 2}\\
	b^{[1]}_{-1} &= \vect{1 \\ 0} ~~ &b^{[2]}_{1} &= \vect{0.25 \\ 2.5}\\
	b^{[2]}_{0} &= \vect{0.25 \\ 1.5} ~~ &b^{[3]}_{1} &= \vect{0.25 \\ 2}
\end{align}

~\vspace{10pt}
%!TEX root = main.tex

\begin{centering}

\psset{linewidth=1.2\pslinewidth}
\psset{xAxisLabel={x},
       yAxisLabel={y},
       labelFontSize=\displaystyle,
       xticksize=-2 6,
       yticksize=-1 5,
       ticklinestyle=dashed
	}
\begin{psgraph}[Ox=0, Oy=0, Dy=1,dy=1,dx=1,Dx=1]{->}(0,0)(-1,-2)(5,6){6cm}{8cm}
\end{psgraph}

\end{centering}

%      \psaxes[labels=none,Ox=1,subticks=5,ticksize=0pt -4pt](3,5)(1,0)(3,5)



% subsection b_ (end)
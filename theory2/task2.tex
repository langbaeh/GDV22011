%!TEX root = main.tex

\section*{Aufgabe 2 Approximation mit Bernstein-Polynomen}

\subsection*{a)}
Eine Bezierkurve von Grad 2 hat 3 Kontrollpunke und 3 Bernstein-Polynome:
\begin{eqnarray}
    B_{20}&=& \frac{2!}{2!}\left(1-t \right)^2 = \left(1-t \right)^2\\
    B_{11}&=& \frac{2!}{1!1!}\left(1-t \right)t = 2 \left(1-t \right)t  \\
    B_{02}&=& \frac{2!}{2!} t^2 = t^2  
\end{eqnarray}




\subsection*{b)}

$\mathbf{d}_k = \sum_{i+j=q} B_{ij}(t_k) \mathbf{b}_{ij}$

$\vect{ \vect{0 \\ 0 } \\ \vect{4 \\ 4 } \\ \vect{8 \\ 2 } \\ \vect{12 \\ 0 }} = 
\left( \begin{matrix} 
    B_{20}(0) & B_{11}(0) &B_{02}(0) \\
    B_{20}(1/3) & B_{11}(1/3) &B_{02}(1/3) \\
    B_{20}(2/3) & B_{11}(2/3) &B_{02}(2/3) \\
    B_{20}(1) & B_{11}(1) &B_{02}(1) \\
\end{matrix} \right)  \vect{ \mathbf{b}_{20} \\ \mathbf{b}_{11} \\ \mathbf{b}_{02} }$


$\vect{ \vect{0 \\ 0 } \\ \vect{4 \\ 4 } \\ \vect{8 \\ 2 } \\ \vect{12 \\ 0 }} = 
\left( \begin{matrix} 
    1 & 2 &0 \\
    4/9 & 4/9 &1/9 \\
    1/9 & 4/9 & 4/9 \\
    0 & 2 &1  \\
\end{matrix} \right)  \vect{ \mathbf{b}_{20} \\ \mathbf{b}_{11} \\ \mathbf{b}_{02} }$

Dies ist ein \"uberbestimmtes Gleichungssystem und es l\"asst sich nicht l\"osen, da die Zeilenvektoren alle linear unabh\"angig voneinander sind.

\subsection*{c)}
$\mathbf{B} ^* = \mathbf{B}^T \mathbf{B} =  \left( \begin{matrix} 
    1 & 2 &0 \\
    4/9 & 4/9 &1/9 \\
    1/9 & 4/9 & 4/9 \\
    0 & 2 &1  \\
\end{matrix} \right) ^T     \left( \begin{matrix} 
        1 & 2 &0 \\
        4/9 & 4/9 &1/9 \\
        1/9 & 4/9 & 4/9 \\
        0 & 2 &1  \\
    \end{matrix} \right) =     1/81\left( \begin{matrix} 
            98 & 182 &8 \\
            182 & 680 &182 \\
            8 & 182 & 98 \\
        \end{matrix} \right) $ 
        
$\mathbf{d}^* = B^T \mathbf{d} = \vect{ \vect{8/3 \\ 2 } \\ \vect{ 88/3 \\ 8/3} \\ \vect{16 \\ 4/3 }} $

$b = \left(\mathbf{B*} \right)^{-1} \mathbf{d^*} = \vect{ \vect{8 \\ 9.3  } \\ \vect{ -4  \\ -4.5} \\ \vect{20 \\ 8.7 }} $

%!TEX root = main.tex

\begin{centering}

\psset{linewidth=1.2\pslinewidth}
\psset{xAxisLabel={x},
       yAxisLabel={y},
       labelFontSize=\displaystyle,
       xticksize=-1 9,
       yticksize=-1 9,
       ticklinestyle=dashed
	}
\begin{psgraph}[Ox=0, Oy=0, Dy=1,dy=1,dx=1,Dx=1]{->}(0,0)(-1,-1)(16,16){10cm}{10cm}
\end{psgraph}

\end{centering}

%      \psaxes[labels=none,Ox=1,subticks=5,ticksize=0pt -4pt](3,5)(1,0)(3,5)



%!TEX root = main.tex

\section*{Bernstein-Bezi\'er-Tensorprodukte}
%-----------------------------------------------
\subsection*{a)} 


Da das Tensorprodukt die Eckpunkte interpoliert sind die Koeffizienten direkt gegeben:
\begin{align}
	P(a,c) &= P(0,0) = 0 = b_{10,10} \\
	P(b,c) &= P(1,0) = 0 = b_{01,10} \\
	P(a,d) &= P(0,1) = 0 = b_{10,01} \\
	P(b,d) &= P(1,1) = 1 = b_{01,01} \\
\end{align}

%-----------------------------------------------

\subsection*{b)}
Fuer den Unterteilungsschritt muessen wir 16 neue Kontrollpunkte berechnen. Wir bestimmen diese indem wir de Casteljeau fuer fixe $u$ bzw. $v$ anwenden und erhalten so:\\
Fuer fixe $v_0$:
\begin{align}
	&P_{20}(\frac{1}{4}) = \frac{1}{4} 
	&P_{20}(\frac{3}{4}) = \frac{1}{4} \\
	&P_{11}(\frac{1}{4}) = \frac{3}{4} 
	&P_{11}(\frac{3}{4}) = \frac{3}{4} \\
	&P_{02}(\frac{1}{4}) = \frac{1}{4} 	
	&P_{02}(\frac{3}{4}) = \frac{1}{4} 
\end{align}
Fuer fixe $u_0$:
\begin{align}
	&P_{20}(\frac{1}{4}) = \frac{1}{4} 
	&P_{20}(\frac{3}{4}) = \frac{1}{4} \\
	&P_{11}(\frac{1}{4}) = \frac{3}{4} 
	&P_{11}(\frac{3}{4}) = \frac{3}{4} \\
	&P_{02}(\frac{1}{4}) = \frac{1}{4} 	
	&P_{02}(\frac{3}{4}) = \frac{1}{4} 
\end{align}
Die restlichen 4 Punkte folgen aus einem weiteren Schritt des de Casteljau:
\begin{align}
	&P(\frac{1}{4},\frac{1}{4}) = \frac{2}{3} 
	&P(\frac{1}{4},\frac{3}{4}) = \frac{2}{3} \\
	&P(\frac{3}{4},\frac{1}{4}) = \frac{2}{3} 
	&P(\frac{3}{4},\frac{3}{4}) = \frac{2}{3} 
\end{align}

% evtl plot
\begin{pspicture}(-3,-3)(3,3) 
    \psline[linewidth=0.5 pt]{*-*}(0,0)(0,-3) 
    \psline[linewidth=0.5 pt]{*-*}(0,0)(0,3) 
    \psline[linewidth=0.5 pt]{*-*}(0,0)(-3,0) 
    \psline[linewidth=0.5 pt]{*-*}(0,0)(3,0)     
    \psline[linewidth=0.5 pt]{*-*}(-3,-3)(-3,0) 
    \psline[linewidth=0.5 pt]{*-*}(-3,0)(-3,3) 
    \psline[linewidth=0.5 pt]{*-*}(3,-3)(3,0) 
    \psline[linewidth=0.5 pt]{*-*}(3,0)(3,3) 
    \psline[linewidth=0.5 pt]{*-*}(3,-3)(0,-3) 
    \psline[linewidth=0.5 pt]{*-*}(0,-3)(-3,-3) 
    \psline[linewidth=0.5 pt]{*-*}(3,3)(0,3) 
    \psline[linewidth=0.5 pt]{*-*}(0,3)(-3,3)
    \uput[225](-3,3){$b_{20,02}$}
    \uput[225](0,3){$b_{11,02}$}
    \uput[225](3,3){$b_{02,02}$}
    \uput[225](-3,0){$b_{20,11}$}
    \uput[225](0,0){$b_{11,11}$}
    \uput[225](3,0){$b_{02,11}$}
    \uput[225](-3,-3){$b_{20,20}$}
    \uput[225](0,-3){$b_{11,20}$}
    \uput[225](3,-3){$b_{02,20}$}
\end{pspicture}

%-----------------------------------------------
\subsection*{c)}


\begin{pspicture}(-7,-7)(7,7) 
    \psdots*[dotstyle=o](-6,-6)(-6,-2)(-6,2)(-6,6) 
    \psdots*[dotstyle=o](-2,-6)(-2,-2)(-2,2)(-2,6) 
    \psdots*[dotstyle=o](2,-6)(2,-2)(2,2)(2,6) 
    \psdots*[dotstyle=o](6,-6)(6,-2)(6,2)(6,6)
    \psdots*[dotstyle=*](0,0)
\end{pspicture}


%-----------------------------------------------

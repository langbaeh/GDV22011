%!TEX root = main.tex

\section*{Bernstein-Bezi\'er-Tensorprodukte}
%-----------------------------------------------
\subsection*{a)} 


Da das Tensorprodukt die Eckpunkte interpoliert sind die Koeffizienten direkt gegeben:
\begin{align}
	P(a,c) &= P(0,0) = 0 = b_{10,10} \\
	P(b,c) &= P(1,0) = 0 = b_{01,10} \\
	P(a,d) &= P(0,1) = 0 = b_{10,01} \\
	P(b,d) &= P(1,1) = 1 = b_{01,01} \\
\end{align}

%-----------------------------------------------

\subsection*{b)}
Fuer den Unterteilungsschritt muessen wir 16 neue Kontrollpunkte berechnen. Wir bestimmen diese indem wir de Casteljeau fuer fixe $u$ bzw. $v$ anwenden und erhalten so:\\
Fuer fixe $v_0$:
\begin{align}
	&P_{20}(\frac{1}{4}) = \frac{1}{4} 
	&P_{20}(\frac{3}{4}) = \frac{1}{4} \\
	&P_{11}(\frac{1}{4}) = \frac{3}{4} 
	&P_{11}(\frac{3}{4}) = \frac{3}{4} \\
	&P_{02}(\frac{1}{4}) = \frac{1}{4} 	
	&P_{02}(\frac{3}{4}) = \frac{1}{4} 
\end{align}
Fuer fixe $u_0$:
\begin{align}
	&P_{20}(\frac{1}{4}) = \frac{1}{4} 
	&P_{20}(\frac{3}{4}) = \frac{1}{4} \\
	&P_{11}(\frac{1}{4}) = \frac{3}{4} 
	&P_{11}(\frac{3}{4}) = \frac{3}{4} \\
	&P_{02}(\frac{1}{4}) = \frac{1}{4} 	
	&P_{02}(\frac{3}{4}) = \frac{1}{4} 
\end{align}
Die restlichen 4 Punkte folgen aus einem weiteren Schritt des de Casteljau:
\begin{align}
	&P(\frac{1}{4},\frac{1}{4}) = \frac{2}{3} 
	&P(\frac{1}{4},\frac{3}{4}) = \frac{2}{3} \\
	&P(\frac{3}{4},\frac{1}{4}) = \frac{2}{3} 
	&P(\frac{3}{4},\frac{3}{4}) = \frac{2}{3} 
\end{align}
Im jeweils ersten Schritt erhalten wir $\square$ und im zweiten Schritt $*$ \\
\\
\begin{center}
\begin{pspicture}(-4,-4)(4,4) 
    \psline[linewidth=0.5 pt]{*-*}(0,0)(0,-4) 
    \psline[linewidth=0.5 pt]{*-*}(0,0)(0,4) 
    \psline[linewidth=0.5 pt]{*-*}(0,0)(-4,0) 
    \psline[linewidth=0.5 pt]{*-*}(0,0)(4,0)     
    \psline[linewidth=0.5 pt]{*-*}(-4,-4)(-4,0) 
    \psline[linewidth=0.5 pt]{*-*}(-4,0)(-4,4) 
    \psline[linewidth=0.5 pt]{*-*}(4,-4)(4,0) 
    \psline[linewidth=0.5 pt]{*-*}(4,0)(4,4) 
    \psline[linewidth=0.5 pt]{*-*}(4,-4)(0,-4) 
    \psline[linewidth=0.5 pt]{*-*}(0,-4)(-4,-4) 
    \psline[linewidth=0.5 pt]{*-*}(4,4)(0,4) 
    \psline[linewidth=0.5 pt]{*-*}(0,4)(-4,4)
    \uput[225](-4,4){$b_{20,02}$}
    \uput[225](0,4){$b_{11,02}$}
    \uput[225](4,4){$b_{02,02}$}
    \uput[225](-4,0){$b_{20,11}$}
    \uput[225](0,0){$b_{11,11}$}
    \uput[225](4,0){$b_{02,11}$}
    \uput[225](-4,-4){$b_{20,20}$}
    \uput[225](0,-4){$b_{11,20}$}
    \uput[225](4,-4){$b_{02,20}$}
	\psdots[linewidth= 2pt, dotstyle=square](-4,-2)(-4,2)(-2,-4)(-2,0)(-2,4)
										(0,-2)(0,2)
										(4,-2)(4,2)(2,-4)(2,0)(2,4)
										
	\psdots[linewidth= 2pt, dotstyle=asterisk](-2,-2)(-2,2)(2,-2)(2,2)
\end{pspicture}
\end{center}

%-----------------------------------------------
\subsection*{c)}
Punkt:\\
\\
\begin{pspicture}(-2,-2)(1,1)
	\pscircle[linestyle=none,fillstyle=solid,fillcolor=red](0,0){0.15}
	\uput[225](0,0){$1$}
	\psline[linewidth=0.5 pt]{*-*}(-1,-1)(-1,0) 
	\psline[linewidth=0.5 pt]{*-*}(-1,0)(-1,1) 
	\psline[linewidth=0.5 pt]{*-*}(0,-1)(0,0) 
	\psline[linewidth=0.5 pt]{*-*}(0,0) (0,1) 
	\psline[linewidth=0.5 pt]{*-*}(1,-1)(1,0) 
	\psline[linewidth=0.5 pt]{*-*}(1,0) (1,1) 
	\psline[linewidth=0.5 pt]{*-*}(-1,-1)(0,-1) 
	\psline[linewidth=0.5 pt]{*-*}(0,-1)(1,-1) 
	\psline[linewidth=0.5 pt]{*-*}(-1,0)(0,0) 
	\psline[linewidth=0.5 pt]{*-*}(0,0)(1,0) 
	\psline[linewidth=0.5 pt]{*-*}(-1,1)(0,1) 
	\psline[linewidth=0.5 pt]{*-*}(0,1)(1,1)
\end{pspicture}
\\ \\
Kante:\\
\\
\begin{pspicture}(-4,-4)(3,1)
	\pscircle[linestyle=none,fillstyle=solid,fillcolor=red](0,-1){0.15}
	\uput[225](-3,-1){$-\frac{1}{16}$}
	\uput[225](-1,-1){$\frac{9}{16}$}
	\uput[225](1,-1){$\frac{9}{16}$}
	\uput[225](3,-1){$-\frac{1}{16}$}
	\multido{\ra=-3+2} {4} {\psline[linewidth=0.5pt]{*-*}(\ra,-3)(\ra,-1)}
	\multido{\ra=-3+2} {4} {\psline[linewidth=0.5pt]{*-*}(\ra,-1)(\ra,1)}
	
	\multido{\ra=-3+2} {3} {\psline[linewidth=0.5pt]{*-*}(-3,\ra)(-1,\ra)}
	\multido{\ra=-3+2} {3} {\psline[linewidth=0.5pt]{*-*}(-1,\ra)(1,\ra)}
	\multido{\ra=-3+2} {3} {\psline[linewidth=0.5pt]{*-*}(1,\ra)(3,\ra)}
	\multido{\ra=-3+2} {3} {\psline[linewidth=0.5pt]{*-*}(-3,\ra)(-1,\ra)}

\end{pspicture}
\\
Flaeche:\\
\\
\begin{pspicture}(-4,-4)(3,3)
	\pscircle[linestyle=none,fillstyle=solid,fillcolor=red](0,0){0.15}
	\uput[225](-3,-3){$\frac{1}{256}$}
	\uput[225](-3,3){$\frac{1}{256}$}
	\uput[225](3,-3){$\frac{1}{256}$}
	\uput[225](3,3){$\frac{1}{256}$}
	
	\uput[225](-3,1){$\frac{-9}{256}$}
	\uput[225](-3,-1){$\frac{-9}{256}$}
	\uput[225](-1,-3){$\frac{-9}{256}$}
	\uput[225](-1,3){$\frac{-9}{256}$}
	\uput[225](1,-3){$\frac{-9}{256}$}
	\uput[225](1,3){$\frac{-9}{256}$}
	\uput[225](3,-1){$\frac{-9}{256}$}
	\uput[225](3,1){$\frac{-9}{256}$}
	
	\uput[225](1,1){$\frac{81}{256}$}
	\uput[225](-1,1){$\frac{81}{256}$}
	\uput[225](1,-1){$\frac{81}{256}$}
	\uput[225](-1,-1){$\frac{81}{256}$}
	
	
	\multido{\ra=-3+2} {4} {\psline[linewidth=0.5pt]{*-*}(\ra,-3)(\ra,-1)}
	\multido{\ra=-3+2} {4} {\psline[linewidth=0.5pt]{*-*}(\ra,-1)(\ra,1)}
	\multido{\ra=-3+2} {4} {\psline[linewidth=0.5pt]{*-*}(\ra,1)(\ra,3)}
	
	\multido{\ra=-3+2} {4} {\psline[linewidth=0.5pt]{*-*}(-3,\ra)(-1,\ra)}
	\multido{\ra=-3+2} {4} {\psline[linewidth=0.5pt]{*-*}(-1,\ra)(1,\ra)}
	\multido{\ra=-3+2} {4} {\psline[linewidth=0.5pt]{*-*}(1,\ra)(3,\ra)}
	\multido{\ra=-3+2} {4} {\psline[linewidth=0.5pt]{*-*}(-3,\ra)(-1,\ra)}

\end{pspicture}



% \begin{pspicture}(-2,-2)(2,2) 
%     \psdots*[dotstyle=o](-6,-6)(-6,-2)(-6,2)(-6,6) 
%     \psdots*[dotstyle=o](-2,-6)(-2,-2)(-2,2)(-2,6) 
%     \psdots*[dotstyle=o](2,-6)(2,-2)(2,2)(2,6) 
%     \psdots*[dotstyle=o](6,-6)(6,-2)(6,2)(6,6)
%     \psdots*[dotstyle=*](0,0)
% \end{pspicture}


%-----------------------------------------------

%!TEX root = main.tex

\section*{Aufgabe 3 B-Splines vom Grad 2}

\subsection*{a)}
Fuer Knoten $x_j$ sind die Basisfunktionen $B^2_{j-1}$ und $B^2_{j-2}$ ungleich $0$. Vereinfachen ergibt:
\begin{eqnarray}
    B^2_{j-1}&=& \frac{x_j - x_{j-1}}{x_{j+1} - x_{j-1}} \\
    B^2_{j-2}&=& \frac{x_{j+1} - x_{j}}{x_{j+1} - x_{j-1}}
\end{eqnarray}

Daraus folgt: 
\begin{equation}
    S(x_j) = \frac{x_j - x_{j-1}}{x_{j+1} - x_{j-1}} b_{j-1} +  \frac{x_{j+1} - x_{j}}{x_{j+1} - x_{j-1}} b_{j-2}
\end{equation}



\subsection*{b)}

Fuer einen uniformen Knotenvektor:
\begin{eqnarray}
    S(x_j) &=& \frac{x_j - x_{j-1}}{x_{j+1} - x_{j-1}} b_{j-1} +  \frac{x_{j+1} - x_{j}}{x_{j+1} - x_{j-1}} b_{j-2} \\
     &=& \frac{\triangle x}{2 \triangle x} b_{j-1} +  \frac{\triangle x}{2 \triangle x} b_{j-2} \\
      &=& 0.5 ( b_{j-1} + b_{j-2}) 
\end{eqnarray}

Bei einem uniformen Knotenvektor geht der Spline $S$ an der Stelle $x_j$ genau durch den Mittelpunkt zwischen $b_{j-1}$ und  $b_{j-2}$.

GUT GENUG ERKLAERT???


\subsection{c)}

Fuer zwei Knoten eines Splines $S$, $x_i$ und $x_{i+1}$ ergibt sich der  naechste Knoten $x_{i+2} = 1/3 *( x_{i+1}- x_i)$.

REICHT DAS ??? 
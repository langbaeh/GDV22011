%!TEX root = main.tex

\section*{Aufgabe 3 B-Splines vom Grad 2}

\subsection*{a)}
Fuer Knoten $x_j$ sind die Basisfunktionen $B^2_{j-1}$ und $B^2_{j-2}$ ungleich $0$. Vereinfachen ergibt:
\begin{eqnarray}
    B^2_{j-1}&=& \frac{x_j - x_{j-1}}{x_{j+1} - x_{j-1}} \\
    B^2_{j-2}&=& \frac{x_{j+1} - x_{j}}{x_{j+1} - x_{j-1}}
\end{eqnarray}

Daraus folgt: 
\begin{equation}
    S(x_j) = \frac{x_j - x_{j-1}}{x_{j+1} - x_{j-1}} b_{j-1} +  \frac{x_{j+1} - x_{j}}{x_{j+1} - x_{j-1}} b_{j-2}
\end{equation}



\subsection*{b)}

Fuer einen uniformen Knotenvektor:
\begin{eqnarray}
    S(x_j) &=& \frac{x_j - x_{j-1}}{x_{j+1} - x_{j-1}} b_{j-1} +  \frac{x_{j+1} - x_{j}}{x_{j+1} - x_{j-1}} b_{j-2} \\
     &=& \frac{\triangle x}{2 \triangle x} b_{j-1} +  \frac{\triangle x}{2 \triangle x} b_{j-2} \\
      &=& 0.5 ( b_{j-1} + b_{j-2}) 
\end{eqnarray}

Bei einem uniformen Knotenvektor geht der Spline $S$ an der Stelle $x_j$ genau durch den Mittelpunkt zwischen $b_{j-1}$ und  $b_{j-2}$.




\subsection*{c)}
Fuer zwei Knoten eines Splines $S$, $x_i$ und $x_{i+1}$ ergibt sich der  naechste Knoten $x_{i+2} = x_{i+1} + 1/3 *( x_{i+1} - x_i)$.

\subsection*{d)}

\begin{eqnarray}
    B^2_{j-1}(x_j) &=& \frac{t - x_{j-1}}{x_{j+1} - x_{j-1}} \\
    B'^2_{j-1}(x_j)&=& \frac{1}{x_{j+1} - x_{j-1}}\\
    &=& \frac{1}{2 \triangle x}
\end{eqnarray}

\begin{eqnarray}
    B^2_{j-2}(x_j)&=& \frac{x_{j+1} - x_{j}}{x_{j+1} - x_{j-1}}\\
    B'^2_{j-2}(x_j)&=& -\frac{1}{x_{j+1} - x_{j-1}}\\
        &=& -\frac{1}{2 \triangle x}
\end{eqnarray}

Damit ist die komplette Ableitung:

\begin{equation}
    S'(x_j) =  \frac{1}{2 \triangle x} b_{j-1} - \frac{1}{2 \triangle x} b_{j-2}
\end{equation}

Die Tangente an $x_j$ ist der Vektor von $b_{j-1}$ nach $b_{j-2}$.

GENUG GELABERT?


\subsection*{e)}

%!TEX root = main.tex

\begin{centering}

\psset{linewidth=1.2\pslinewidth}
\psset{xAxisLabel={x},
       yAxisLabel={y},
       labelFontSize=\displaystyle,
       xticksize=-0.5 4.2,
       yticksize=-0.5 9.2,
       ticklinestyle=dashed
	}
\begin{psgraph}[Ox=0, Oy=0, Dy=1,dy=1,dx=1,Dx=1]{->}(0,0)(-1,-1)(19,9){10cm}{5cm}
\end{psgraph}

\end{centering}

%      \psaxes[labels=none,Ox=1,subticks=5,ticksize=0pt -4pt](3,5)(1,0)(3,5)



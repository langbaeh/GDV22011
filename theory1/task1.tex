%!TEX root = main.tex

\section{Aufgabe 1 Darstellungsformen}

\subsection*{a)}


\begin{equation}
    {\bf P}_M(t) = a_0 + a_1 t + a_2 t^2
\end{equation}

Find ${\bf P}_M$ through inversion of the Vandermonde Matrix:

\begin{equation}
    \vect{1 & 0 & 0 \\ 1 & 1 & 1 \\ 1 & 2 & 4} \cdot \vect{a_0 \\ a_1 \\ a_2} = \vect{P_0 \\ P_1 \\ P_2}
\end{equation}

Inverting yields:

\begin{equation}
    \vect{1 & 0 & 0 \\ -3/2 & 2  & -1/2  \\ 1/2 & -1 & 1/2} \cdot \vect{P_0 \\ P_1 \\ P_2} = \vect{a_0 \\ a_1 \\ a_2} 
    = \vect{P_0 \\ -3/2 P_0 - 1/2 P_2 + 2 P_1 \\ 1/2 P_0 - P_1 + 1/2 P_2}
\end{equation}


\begin{eqnarray}
     {\bf P}_M(t) &=&  P_0  + \left( -3/2 P_0 - 1/2 P_2 + 2 P_1 \right) t + \left( 1/2 P_0 - P_1 + 1/2 P_2 \right) t^2 \\
&=& \vect{0 \\0} + \vect{0.5 \\ 5.5 } t   + \vect{0.5 \\ - 2.5 } t^2 
\end{eqnarray}


TODO: KURVE AUSWERTEN

%!TEX root = main.tex

\begin{centering}

\psset{linewidth=1.2\pslinewidth}
\psset{xAxisLabel={x},
       yAxisLabel={y},
       labelFontSize=\displaystyle,
       xticksize=-5 5,
       yticksize=0 5,
       ticklinestyle=dashed
	}
\begin{psgraph}[Ox=0, Oy=0, Dy=1,dy=1,dx=1,Dx=1]{->}(0,0)(0,-8)(8,8){5cm}{10cm}
\end{psgraph}

\end{centering}

%      \psaxes[labels=none,Ox=1,subticks=5,ticksize=0pt -4pt](3,5)(1,0)(3,5)




\subsection*{b)}



\begin{equation}
    {\bf P}_L(t) = l_0(t) P_0 + l_1(t) P_1 + l_2 (t) P_2
\end{equation}

\begin{eqnarray}
    l_0(t) &=& \frac{(t-1)(t-2)}{2} \\
    l_1(t) &=& -t(t-2) \\
    l_2(t) &=& \frac{(t-1)t}{2}
\end{eqnarray}

\begin{eqnarray}
    {\bf P}_L(t) &=& \frac{(t-1)(t-2)}{2} P_0 - t(t-2) P_1 + \frac{(t-1)t}{2} P_2 \\
    &=& \frac{t^2-t-2t+2}{2} P_0 - (t^2-2t) P_1 + \frac{(t^2-t}{2} P_2  \\
    &=& P_0  + \left( -3/2 P_0 - 1/2 P_2 + 2 P_1 \right) t + \left( 1/2 P_0 - P_1 + 1/2 P_2 \right) t^2  = {\bf P}_M(t)
\end{eqnarray}



\subsection*{c)}


\begin{equation}
    {\bf P}_N(t) = \omega_0(t) \Delta(t_0)  +  \omega_1(t) \Delta(t_0, t_1) + \omega_2(t) \Delta(t_0, t_1, t_2)
\end{equation}


\begin{tabular}{ccc|ccc}
  & \multirow{3}{*}{1} & 0& \vect{0\\0}& \multirow{3}{*}{\vect{1\\3}} &\\ 
2 & \multirow{3}{*}{1} & 1& \vect{1\\3}& \multirow{3}{*}{\vect{2\\-2}} & \vect{0.5\\-2.5} \\ 
  &                    & 2& \vect{3\\1}&  & \\ 
\end{tabular}

\begin{eqnarray}
    {\bf P}_N(t) &=& 1 \cdot \vect{0 \\0}   +  ( t-0) \vect{1 \\ 3} + (t-0)(t-1) \vect{0.5 \\ -2.5 }\\
    &=& \vect{0 \\0} + \vect{0.5 \\ 5.5 } t   + \vect{0.5 \\ - 2.5 } t^2  = {\bf P}_M(t)
\end{eqnarray}


\subsection*{d)}
Bei der Monom Darstellung m\"ussten wir eine neue 4 x 4 Matrix invertieren, bei der Lagrange Darstellung m\"usste man alle $l_i$ neu berechnen. Nur bei der Newton-Darstellung kann man Berechnungen wiederverwenden:

\begin{equation}
    {\bf P}_N(t) = \omega_0(t) \Delta(t_0)  +  \omega_1(t) \Delta(t_0, t_1) + \omega_2(t) \Delta(t_0, t_1, t_2) + \omega_3(t) \Delta(t_0, t_1, t_2, t_3)
\end{equation}


\begin{tabular}{cccc|cccc}
                  &   & \multirow{2}{*}{1} & 0& \vect{0\\0}& \multirow{3}{*}{\vect{1\\3}} & &\\ 
\multirow{3}{*}{3}&2  & \multirow{3}{*}{1} & 1& \vect{1\\3}& \multirow{3}{*}{\vect{2\\-2}} & \vect{0.5\\-2.5} & \multirow{2}{*}{ \vect{ - 5/6 \\ 0.5}}\\ 
                  &2  & \multirow{3}{*}{1} & 2& \vect{3\\1}& \multirow{3}{*}{\vect{0\\-2}} & \vect{-2\\-1} & \\ 
                  &   &                    & 3& \vect{1\\1}&  &  & \\ 
\end{tabular}

\begin{eqnarray}
    {\bf P}_N(t) &=& 1 \cdot \vect{0 \\0}   +  ( t-0) \vect{1 \\ 3} + (t-0)(t-1) \vect{0.5 \\ -2.5 } + (t-0)(t-1)(t-2) \vect{ - 5/6 \\ 0.5} \\
    &=& \vect{0 \\0} + \vect{- 7/6 \\ 5.75 } t   + \vect{ -2  \\ - 4 } t^2  + \vect{ - 5/6 \\ 0.5} t^3 
\end{eqnarray}

